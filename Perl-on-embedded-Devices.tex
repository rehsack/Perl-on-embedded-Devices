\documentclass[ngerman,xcolor={table,dvipsnames},smaller,compress,hyperref={bookmarks,colorlinks},handout]{beamer}%

\usepackage{url}
\usepackage{listings}
\usepackage[latin9]{inputenc}
\usepackage{textcomp}
%\usepackage{auto-pst-pdf}
\usepackage{graphicx}
\usepackage{pstricks}
\usepackage{pst-node}
\usepackage{pst-uml}
\usepackage{pst-tree}
\usepackage{pst-blur}

\usepackage{amsfonts}
\usepackage{amssymb}
\usepackage{amsmath}
\usepackage{pifont}

% for multipage tables (xtab or longtable
\usepackage{longtable}
\usepackage{lscape}
\usepackage{booktabs}
\usepackage[tableposition=top]{caption}

\title{Perl on embedded Devices}
\author{Jens Rehsack}
\institute[Niederrhein.PM]{Niederrhein Perl Mongers}
\date{German Perl Workshop 2018}

\usetheme[secheader]{Boadilla}
\setbeamertemplate{navigation symbols}{}

\newcommand{\perlfilename}[1]{{\color{cyan}{\textit{\begingroup \urlstyle{sf}\Url{#1}}}}}
\newcommand{\xsfilename}[1]{{\color{olive}{\textit{\begingroup \urlstyle{sf}\Url{#1}}}}}
\newcommand{\hfilename}[1]{{\color{olive}{\textit{\begingroup \urlstyle{sf}\Url{#1}}}}}
\newcommand{\cfilename}[1]{{\color{magenta}{\textit{\begingroup \urlstyle{sf}\Url{#1}}}}}
\newcommand{\variable}[1]{{\color{violet}{\textsf{#1}}}}
\newcommand{\variablew}[1]{{{\textsf{#1}}}}
\newcommand{\command}[1]{'\texttt{#1}'}

\def\signed #1{{\leavevmode\unskip\nobreak\hfil\penalty50\hskip2em
   \hbox{}\nobreak\hfil(#1)%
   \parfillskip=0pt \finalhyphendemerits=0 \endgraf}}
 
\newsavebox\mybox
\newenvironment{aquote}[1]
   {\savebox\mybox{#1}\begin{quotation}}
   {\signed{\usebox\mybox}\end{quotation}}

\makeatletter
\makeatother

\begin{document}

\psset{angleA=90,angleB=-90}
\lstset{language=Perl,
        basicstyle=\ttfamily,
        keywordstyle=\color{Maroon},
        commentstyle=\color{Blue}, 
        stringstyle=\color{Green},
        showstringspaces=false}

\AtBeginPart{\begin{frame}<beamer> \frametitle{Overview} \partpage \tableofcontents[current] \end{frame}}

\frame{\maketitle}

\part{Setup}

\begin{frame}[fragile]{Login on Build Server}
\begin{block}<1->{Login on Build Server}
\begin{lstlisting}[language=sh,inputencoding=latin9,escapeinside={(*@}{@*)}]
$ ssh -Y gpw_acc_${VLAN_GRP}@10.0.${VLAN_GRP}0.1
# e.g. ssh -Y gpc_acc_4@10.0.40.1
\end{lstlisting}
\end{block}

Your password is identical to your login name.

\begin{block}<2->{Attach your device (to your Laptop)}
\begin{lstlisting}[language=sh,inputencoding=latin9,escapeinside={(*@}{@*)}]
$ ls -l /dev | grep -i usb
$ screen /dev/tty.usbserial 115200
\end{lstlisting}
\end{block}

\uncover<2->{On Windows use \texttt{putty}, for example.}

\uncover<3->{Optionally power on your device and watch the boot process}

\end{frame}

\begin{frame}[fragile]{Setup Build-Tree}
\begin{block}<1->{Prepare Build-Tree (Build Server)}
\scriptsize
\begin{lstlisting}[language=sh,inputencoding=latin9,escapeinside={(*@}{@*)}]
$ mkdir -p ~/gpw-community-bsp/sources
$ cd ~/gpw-community-bsp/
$ git clone -b master https://github.com/rehsack/gpw-yocto-platform.git
$ cd sources
$ git clone -b master https://github.com/rehsack/poky.git
$ git clone -b master https://github.com/rehsack/meta-openembedded.git
$ git clone -b master https://github.com/meta-cpan/meta-cpan.git
$ git clone -b master https://github.com/rehsack/meta-jens.git
$ git clone -b master https://github.com/rehsack/meta-gpw.git
\end{lstlisting}
\end{block}
\end{frame}

\begin{frame}[fragile]{Setup Environment}
\begin{block}<1->{Prepare Environment (Build Server)}
\scriptsize
\begin{lstlisting}[language=sh,inputencoding=latin9,escapeinside={(*@}{@*)}]
# gpw_acc_5 does, gpw_acc_8 uses 8
$ echo "export VLAN_GRP=5
export LANG=en_US.UTF-8
source ~/gpw-community-bsp/sources/poky/scripts/oe-init-bashrc" >> ~/.bashrc
\end{lstlisting}
\end{block}

Please use your VLAN group from your name plate on the desk.

\begin{block}<2->{Restart and prepare your shell (Build Server)}
\begin{lstlisting}[language=sh,inputencoding=latin9,escapeinside={(*@}{@*)}]
$ exec bash
$ oe_builddir use ~/gpw-community-bsp/gpw-yocto-platform
\end{lstlisting}
\end{block}
\end{frame}

\begin{frame}[fragile]{Setup Cache}
\begin{block}<1->{Prepare Cache (Build Server)}

Add the following lines to \$BUILDDIR/conf/local.conf

\begin{lstlisting}[language=sh,inputencoding=latin9,escapeinside={(*@}{@*)}]
BB_GENERATE_MIRROR_TARBALLS = "1"
INHERIT += "own-mirrors"
SOURCE_MIRROR_URL = "file:///home/gpw2018/downloads"
SSTATE_MIRRORS ?= "\
  file://.* file:///home/gpw2018/sstate-cache/PATH"
\end{lstlisting}
\end{block}
\end{frame}

\part{First Steps}

\begin{frame}[fragile]{Enter Buildmode (Build Server)}
\begin{block}<1->{Enter Buildmode}
\small
\begin{lstlisting}[language=sh,inputencoding=latin9,escapeinside={(*@}{@*)}]
# create new screen session named bitbake
$ screen -S bitbake
$ oe_builddir use ~/gpw-community-bsp/gpw-yocto-platform
$ WANTED_ROOT_DEV=emmc mkimgs updatable-app-image
\end{lstlisting}
\end{block}
\end{frame}

\begin{frame}[fragile]{Enter Flash (Build Server)}
\begin{block}<1->{Enter Flash (Build Server)}
\begin{lstlisting}[language=sh,inputencoding=latin9,escapeinside={(*@}{@*)}]
# re-connect to screen session named bitbake
$ screen -r bitbake
$ WANTED_ROOT_DEV=emmc gpw_deploy flash
\end{lstlisting}
\end{block}

\begin{block}<2->{Reset your device (Your Laptop)}
Reset your device and stop automatic booting by pressing space on the terminal
attached to the device serial console.
\end{block}

\begin{block}<3->{Flash device (Your Laptop, Device Screen)}
\begin{lstlisting}[language=sh,inputencoding=latin9,escapeinside={(*@}{@*)}]
run netboot
\end{lstlisting}
\end{block}
\end{frame}

\part{On your own}

\begin{frame}[fragile]{Add packages (Build Server)}
\begin{block}<1->{Add rakudo*}
\small
\begin{lstlisting}[language=sh,inputencoding=latin9,escapeinside={(*@}{@*)}]
$ cd ${BSPDIR}/sources/meta-gpw
$ vim recipes-images/images/updatable-app-image.bbappend
IMAGE_INSTALL_append = "\
  rakudo-star \
"
\end{lstlisting}
\end{block}

\begin{block}<2->{Add Dancer2 (Build Server)}
\small
\begin{lstlisting}[language=sh,inputencoding=latin9,escapeinside={(*@}{@*)}]
$ cd ${BSPDIR}/sources/meta-gpw
$ vim recipes-images/images/updatable-app-image.bbappend
IMAGE_INSTALL_append = "\
  dancer2-perl \
"
\end{lstlisting}
\end{block}
\end{frame}

\part{Into the future}

\begin{frame}[fragile]{Update image}
\begin{block}<1->{Update from Upstream (Build Server)}
\small
\begin{lstlisting}[language=sh,inputencoding=latin9,escapeinside={(*@}{@*)}]
$ cd ${BSPDIR}/sources/poky
$ git remote add upstream git://git.yoctoproject.org/poky
$ git fetch upstream
$ git rebase upstream/master
\end{lstlisting}
\end{block}

Repeat this for all cloned repositories \ldots
\end{frame}

\begin{frame}[fragile]{Own software}
\begin{block}<1->{Add self written software (Build Server)}
\scriptsize
\begin{lstlisting}[language=sh,inputencoding=latin9,escapeinside={(*@}{@*)}]
$ cd ${BSPDIR}/sources/meta-gpw
$ mkdir recipes-service/myservice
$ vim recipes-service/myservice/myservice_git.bb
DESCRIPTION = "My::Service provides backend for cool stuff"
LICENSE     = "MIT"
MAINTAINER  = "my.name@my.company"
HOMEPAGE    = "https://myservice.my.company/"
LIC_FILES_CHKSUM = "file://${COMMON_LICENSE_DIR}/MIT;md5=..."

SRC_REV = "abcdef0123456789abcdef0123456789abcdef01"
SRC_URI = "\
  git://scm.my.company/path/repo.git;rev=${SRC_REV} \
  file://run \
  file://my-service.json \
  ..."

RDEPENDS_${PN} += "daemontools"
RDEPENDS_${PN} += "dancer2"
RDEPENDS_${PN} += "moox-configfromfile-perl"

do_install_appends() {
    install -d -m ${D}${sysconfdir}
    install -m 0644 ${WORKDIR}/my-service.json ${D}${sysconfdir}
    install -m 0755 ${WORKDIR}/run ${D}${sysconfdir}/daemontools/service/run
}
\end{lstlisting}
\end{block}
\end{frame}

\begin{frame}[fragile]{Prepare Packger::Utils}
\begin{block}<1->{Prepare Packger::Utils (Build Server)}
\small
\begin{lstlisting}[language=sh,inputencoding=latin9,escapeinside={(*@}{@*)}]
# checkout and prepare Packager::Utils
$ mkdir ~/Projects
$ cd ~/Projects
$ git clone https://github.com/rehsack/Packager-Utils.git
$ cd Packager-Utils
$ perl Makefile.PM
$ cpanm installdeps .
\end{lstlisting}
\end{block}
\end{frame}

\begin{frame}[fragile]{Recipes for CPAN modules}
\begin{block}<1->{Recipes for CPAN modules (Build Server)}
\small
\begin{lstlisting}[language=sh,inputencoding=latin9,escapeinside={(*@}{@*)}]
# create list of available perl/cpan packages
$ make && env BBPATH=$BUILDDIR perl -Mblib bin/pkg_util \
  check up2date --packages_pattern "*-perl_*.bb" --packages_pattern "perl*.bb" -o log
$ make && env BBPATH=$BUILDDIR perl -Mblib bin/pkg_util \
  create package --categories www --modules Mojolicious
\end{lstlisting}
\end{block}
\end{frame}

\end{document}
